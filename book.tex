\documentclass{book}

\usepackage[english]{babel}
\usepackage[utf8x]{inputenc}
\usepackage{amsmath}
\usepackage{amssymb}
\usepackage{amsfonts}
\usepackage{graphicx}
\usepackage[ruled]{algorithm2e}
\usepackage{empheq}
\usepackage{float}

\title{Algorithmic Programming in Java}
\author{Rodion Efremov} 

\begin{document}
\maketitle

\part*{Preface}
This book is aiming to provide insight into practical implementation of algorithms and data structures in a way that is as accessible to readers of all proficiency levels. Only very basic knowledge of Java programming language and other tools is assumed; wherever relevant, the new language features are explained succinctly without further ado. The book is organized in a ``bottom-up'' fashion so that the book is better read linearly, but not necessarily continuously whenever some topics do not interest. There is no exercises. Instead, the book presents the code listings for a Java library of algorithms and data structures, so the book might be used as a collection of ``algorithmic recipes''. 

The second objective is to provide a gentle introduction to formalism taking place in computer science independent of particular tool sets. By formalism we mean most often pseudo-code, set notation, sequences and functions. We, however, will not do any fancy mathematics with Greek letters such as proving theorems, etc. Whether to learn using formalism or not mainly depend on your academic aims: in case you aim to enroll in a department of computer science, this book just might be everything you need to get prepared for your studies.

\part{Introduction}

\chapter{Coding conventions}

\end{document}